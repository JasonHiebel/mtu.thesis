\documentclass[twoside]{mtu.thesis}

%%% define a header to be used
%%% this example will alternate the mark, even when oneside is chosen
\fancyhead[LE,RO]{\leftmark}

%%% for filler content
\usepackage{mwe}
\usepackage{lipsum}

%%% for a nice looking copyright symbol
\usepackage{textcomp}

%%% paragraph skip and indentation are not rigidly defined
\setlength{\parskip}{1ex plus 0.5ex minus 0.0ex}
\setlength{\parindent}{0pt}

%%% required document information
%%% replace \dissertation with \thesis or \report as appropriate
\title        {Sample Document}
\author       {Jason Hiebel}
\date         {2018}
\copyright    {\raisebox{1pt}{\small\textcopyright} 2018 Some Author}
\dissertation {Your Degree i.e. Doctor of Philosophy}{Degree Program}
\affiliation  {School or Department}

%%% bibliography information
\bibliographystyle{plain}

%%% basic note command for loudly making sure you attend to something
\usepackage{color}
\newcommand{\note}[1]{{\color{red}\emph{#1}}}

%%% the document!
\begin{document}
\frontmatter

% title page
\maketitle

% approval page
\begin{approval}
Dissertation Advisor & My Advisor           \cr
    Committee Member & A Committee Member   \cr
    Committee Member & A Committee Member   \cr
    Committee Member & A Committee Member   \cr
    Department Chair & The Department Chair \cr
\end{approval}

% dedication page (optional)
\begin{dedication}
\emph{To be determined.}
\end{dedication}

% table of contents
\tableofcontents

% list of figures (optional)
\listoffigures

% list of tables (optional)
\listoftables

% preface (required under circumstances)
\begin{preface}
\lipsum[1-2]
\end{preface}

% acknowledgements (optional)
\begin{acknowledgements}
\lipsum[1]
\end{acknowledgements}

% abstract
\begin{abstract}
\lipsum[1-7]
\end{abstract}

\mainmatter

\chapter{Introduction}
\footnotetext[1]{\lipsum[10]}
\lipsum[1-2]

\lipsum[3]

\note{Captions go below a figure and above a table. The classic argument for this use case is that tabular data is a textual device and follows the English reading convention of top-down, left to right reading whereas captions are iconic and are not  instinctually scanned as text.}

\begin{figure}[t]
\centering
\includegraphics[width=.48\linewidth]{example-image-a}\hfill
\includegraphics[width=.48\linewidth]{example-image-b}
\caption[The \emph{mwe} package is pretty great!]{Like the \emph{lipsum} package for text, there is there \emph{mwe} package for images. Enjoy these glorious images in your sample file!}
\end{figure}

\lipsum[4-6]

\begin{table}
\caption{Let's talk about the weather.}
\centering
\begin{tabular}{lrrp{10cm}}
\hline
Day & Low & High & Summary \\\hline
Monday & 11C & 22C & A clear day with lots of sunshine. However, the strong breeze will bring down the temperatures. \\
Tuesday & 9C & 19C & Cloudy with rain, across many northern regions. Clear spells across most of Scotland and Northern Ireland, but rain reaching the far northwest. \\
Wednesday & 10C & 21C & Rain will still linger for the morning. Conditions will improve by early afternoon and continue throughout the evening. \\\hline
\end{tabular}
\end{table}

\lipsum[7-10]

\section{A Section}
\lipsum[10-12]

\subsection{A Subsection}
\lipsum[13]

\section{Another Section}
\lipsum[14-20]

\chapter{Goals and Hypotheses}
\lipsum[1-18]

\chapter{Methods}
\lipsum[1-20]

\chapter{Results}
\lipsum[1-20]

\chapter{Discussion}
\lipsum[1-20]

\nocite{*}
\bibliography{sample}

\appendix

\chapter{Additional Definitions}
\lipsum[1-20]

\chapter{Future Work}
\lipsum[1-5]
\section{Things for an Undergraduate}
\lipsum[1-10]

\end{document}
